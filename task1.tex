\documentclass[11pt]{article}
\usepackage{amsmath,amsthm,amssymb}
\usepackage{cmap}
\usepackage[utf8]{inputenc}
\usepackage[english,russian]{babel}
\usepackage{svg}
\usepackage{listings}
\usepackage{enumitem}
\usepackage{graphicx}
\usepackage{wrapfig}
\usepackage{hyperref}
\usepackage{caption,subcaption}
\usepackage{float}
\usepackage{pdfpages}
\usepackage[nottoc,numbib]{tocbibind}
\usepackage{tabularx}
\usepackage[T1]{fontenc}
\usepackage{tikz}
\usepackage{standalone}


\renewcommand{\baselinestretch}{1.5}
\newcommand{\rulesep}{\unskip\ \vrule\ }

%\renewcommand{\#}{C{\lserif\#}}

\newcommand{\code}[1]{\lstinline$#1$}
%\newcommand{\code}[1]{#1}

\newcommand{\op}[1]{\operatorname{#1}}
\newcommand{\sonop}[4]{#1~\leftarrow~\mathbf{#2}[#3](#4)}
\newcommand{\sonopa}[4]{#1~\leftarrow&~\mathbf{#2}[#3](#4)}

\lstset{captionpos=bellow}
\lstset{basicstyle=\small\ttfamily,showspaces=false,showstringspaces=false}

\theoremstyle{definition}
\newtheorem{definition}{Определение}

\theoremstyle{lemma}
\newtheorem{lemma}{Лемма}

\theoremstyle{statement}
\newtheorem{statement}{Утверждение}

\renewcommand\qedsymbol{QED}

\newlength{\myleftlen}
\newcommand{\setmyleftlen}[1]{\settowidth{\myleftlen}{\( \displaystyle
#1\)}}
\newcommand{\backup}{\hskip-\myleftlen\mkern-7mu}

\tikzset{>=latex}

\title{Решения задач по материалам лекции 1}
\author{Алексей Глушко}

\begin{document}
\maketitle
\section*{Задача 1}
    \subsection*{а)}
        Белая вершина пренадлежит интерпретациям понятий вида:
        \begin{itemize}
            \item $A$, $B$, $A \sqcap B$
            \item $ \{\forall/\exists\} r.(B \sqcap \{\forall/\exists\} r.(A \sqcup B)) $
            \item $ \exists r.((A \sqcup B) \sqcap \{\forall/\exists\} r.(A \sqcup B)) $
        \end{itemize}
        а также их непосредственным следствиям.

    \subsection*{б)}
        \begin{itemize}
            \item $ A \sqsubseteq \{\forall/\exists\} r.(B \sqcap \{\forall/\exists\} r.(A \sqcup B)) $
            \item $ B \sqsubseteq \exists r.(A \sqcup B) $
            \item $ B \sqsubseteq \forall r.B $
            \item \dots
        \end{itemize}

\pagebreak
\section*{Задача 2}
    Искомая модель:
    \begin{figure}[H]
        \centering
        \begin{tikzpicture}[yscale=-1]
            \node[circle, fill] (n) at (0, 0) {};
            \node[circle, draw, label={$A$}] (a) at (2, -1) {};
            \node[circle, draw, label={$\neg A$}] (na) at (2, 1) {};
            \node[circle, draw, label={$X$}] (x1) at (4, -1) {};
            \node[circle, draw, label={$Y$}] (y2) at (4, 1) {};

            \draw[->] (n) to node[midway,above] {r} (a);
            \draw[->] (n) to node[midway,above] {r} (na);

            \draw[->] (a) to node[midway,above] {r} (x1);
            \draw[->] (na) to node[midway,above] {r} (y2);
        \end{tikzpicture}
    \end{figure}

\section*{Задача 3}
    Нормализация:
    \begin{multline*}
        \begin{aligned}
            A \sqsubseteq&~\exists r.(B \sqcap C) \\
            B \sqsubseteq&~D \\
            C \sqsubseteq&~E \\
            D \sqcap E \sqsubseteq&~A \\
            \exists r.\exists r.A \sqsubseteq&~B \\
        \end{aligned}
        \rightarrow
        \begin{aligned}
            \mathbf{A} \sqsubseteq&~\mathbf{\exists r.X} \\
            \mathbf{X} \sqsubseteq&~\mathbf{B \sqcap C} \\
            \mathbf{B \sqcap C} \sqsubseteq&~\mathbf{X} \\
            B \sqsubseteq&~D \\
            C \sqsubseteq&~E \\
            D \sqcap E \sqsubseteq&~A \\
            \exists r.\exists r.A \sqsubseteq&~B \\
        \end{aligned}
        \rightarrow
        \begin{aligned}
            A \sqsubseteq&~\exists r.X \\
            \mathbf{X} \sqsubseteq&~\mathbf{B} \\
            \mathbf{X} \sqsubseteq&~\mathbf{C} \\
            B \sqcap C \sqsubseteq&~X \\
            B \sqsubseteq&~D \\
            C \sqsubseteq&~E \\
            D \sqcap E \sqsubseteq&~A \\
            \exists r.\exists r.A \sqsubseteq&~B \\
        \end{aligned}
        \rightarrow
        \begin{aligned}
            A \sqsubseteq&~\exists r.X \\
            X \sqsubseteq&~B \\
            X \sqsubseteq&~C \\
            B \sqcap C \sqsubseteq&~X \\
            B \sqsubseteq&~D \\
            C \sqsubseteq&~E \\
            D \sqcap E \sqsubseteq&~A \\
            \mathbf{\exists r.Y} \sqsubseteq&~\mathbf{B} \\
            \mathbf{Y} \sqsubseteq&~\mathbf{\exists r.A} \\
            \mathbf{\exists r.A} \sqsubseteq&~\mathbf{Y} \\
        \end{aligned}
    \end{multline*}

    Проверим истинность $\mathcal{O} \models A \sqsubseteq D$.
    \begin{enumerate}
        \item Пусть $R(r) = \varnothing$ и $S(Z) = \{Z\}$ для всех $r$ и $Z$;
        \item $A \in S(A) ~\&~ A \sqsubseteq \exists r.X \implies R(r) \ni \{\langle A, X \rangle\}$;
        \item $Y \in S(Y) ~\&~ Y \sqsubseteq \exists r.A \implies R(r) \ni \{\langle Y, A \rangle\}$;
        \item $\langle A, X \rangle \in R(r) ~\&~ A \in S(X) ~\&~ \exists r.A \sqsubseteq Y \implies S(A) \ni Y$;
        \item $\langle Y, A \rangle \in R(r) ~\&~ Y \in S(A) ~\&~ \exists r.Y \sqsubseteq B \implies S(Y) \ni B$;
        \item $B \in S(Y) ~\&~ Y \in S(A) \implies B \in S(A) $;
        \item $B \in S(A) ~\&~ B \sqsubseteq D \implies S(A) \ni D$;
        \item $D \in S(A) \implies \mathcal{O} \models A \sqsubseteq D$ \\QED.
    \end{enumerate}

\end{document}