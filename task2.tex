\documentclass[11pt]{article}
\usepackage{amsmath,amsthm,amssymb,amsbsy}
\usepackage{cmap}
\usepackage[utf8]{inputenc}
\usepackage[english,russian]{babel}
\usepackage{svg}
\usepackage{listings}
\usepackage{enumitem}
\usepackage{graphicx}
\usepackage{wrapfig}
\usepackage{hyperref}
\usepackage{caption,subcaption}
\usepackage{float}
\usepackage{pdfpages}
\usepackage[nottoc,numbib]{tocbibind}
\usepackage{tabularx}
\usepackage[T1]{fontenc}
\usepackage{tikz}
\usepackage{standalone}
\usepackage{ebproof}


\renewcommand{\baselinestretch}{1.5}
\newcommand{\rulesep}{\unskip\ \vrule\ }

%\renewcommand{\#}{C{\lserif\#}}

\newcommand{\code}[1]{\lstinline$#1$}
%\newcommand{\code}[1]{#1}

\newcommand{\op}[1]{\operatorname{#1}}
\newcommand{\sonop}[4]{#1~\leftarrow~\mathbf{#2}[#3](#4)}
\newcommand{\sonopa}[4]{#1~\leftarrow&~\mathbf{#2}[#3](#4)}

\lstset{captionpos=bellow}
\lstset{basicstyle=\small\ttfamily,showspaces=false,showstringspaces=false}

\theoremstyle{definition}
\newtheorem{definition}{Определение}

\theoremstyle{lemma}
\newtheorem{lemma}{Лемма}

\theoremstyle{statement}
\newtheorem{statement}{Утверждение}

\renewcommand\qedsymbol{QED}

\newlength{\myleftlen}
\newcommand{\setmyleftlen}[1]{\settowidth{\myleftlen}{\( \displaystyle
#1\)}}
\newcommand{\backup}{\hskip-\myleftlen\mkern-7mu}

\tikzset{>=latex}

\newcommand{\hsomet}[2]{\hsome{\textit{#1}}{\textbf{#2}}}
\newcommand{\hsome}[2]{\exists#1.#2}
\newcommand{\honlyt}[2]{\honly{\textit{#1}}{\textbf{#2}}}
\newcommand{\honly}[2]{\forall#1.#2}
\newcommand{\his}{\sqsubseteq}
\newcommand{\hisb}{~{\color{red}\boldsymbol{\sqsubseteq}}~}
\newcommand{\hand}{\sqcap}
\newcommand{\hor}{\sqcup}

\title{Решения задач по материалам лекции 2}
\author{Алексей Глушко}

\begin{document}
\maketitle
\section*{Задача 1}
    \subsection*{а)}
        Онтология:
        \begin{align}
            \hsomet{слушает}{Класика} &\his \honlyt{слушает}{Классика} \label{ax11} \\
            \hsomet{слушает}{Металл} &\his \honlyt{слушает}{Металл} \label{ax12} \\
            \textbf{Классика} &\his \neg \textbf{Металл} \label{ax13} \\
            \textbf{Классический Металлист} &\his \hsomet{слушает}{Классика} \hand \hsomet{слушает}{Металл} \label{ax14}
        \end{align}

    \subsection*{б)}
        \begin{figure}[H]
            \centering
            \begin{tikzpicture}
                \node[circle, fill, label={[align=left]$\hsomet{c}{K} \hand \hsomet{c}{M}$ \\ $\hsomet{c}{K}$ \\ $\hsomet{c}{M}$}] (km) at (0, 0) {};
                \node[circle, fill, label={[align=left]$\textbf{K}$, {\color{red} $\neg \textbf{M}$, $\textbf{M}$}}, color=red] (k) at (4, 0) {};
                \draw[->] (km) to node[midway,below] {\textit{слушает}} (k);
            \end{tikzpicture}
        \end{figure}

    \subsection*{в)}
        \begin{itemize}
            \item $\mathcal{O}_1 = \{\text{\ref{ax11}, \ref{ax13}, \ref{ax14}}\}$
            \item $\mathcal{O}_2 = \{\text{\ref{ax12}, \ref{ax13}, \ref{ax14}}\}$
        \end{itemize}

\pagebreak
\section*{Задача 2}
    \newcommand{\hx}{A \hand \hsome{r}{B}}
    Построим замыкаение $\op{init}(\hx)$ относительно правил прямого вывода.
    \begin{align}
        \text{ax1}:~   & B \hand \hsome{r}{\top} \his C    \tag{ax1} \label{ax1} \\
        \text{ax2}:~   & A \his \hsome{s}{(A \hand B)}     \tag{ax2} \label{ax2} \\
        \text{ax3}:~   & \hsome{s}{\hsome{s}{\top}} \his B \tag{ax3} \label{ax3} \\
                       & \op{init}(\hx)                    \tag{0} \label{x} \\
        R_0(\ref{x}):~                   & \hx \his \hx               \label{t1} \\
        R_\top(\ref{x}):~                & \hx \his \top              \label{t2} \\
        R^-_\hand(\ref{t1}):~            & \hx \his A                 \label{t3} \\
        R^-_\hand(\ref{t1}):~            & \hx \his \hsome{r}{B}      \label{t4} \\
        R_{\op{init}}(\ref{t4}):~        & \op{init}(B)           \label{t5} \\
        R_0(\ref{t5}):~                  & B \his B \\
        R_\his(\ref{ax2}, \ref{t3}):~    & \hx \his \hsome{s}{A \hand B} \label{t6} \\
        R_{\op{init}}(\ref{t6}):~        & \op{init}(A \hand B)      \label{t7} \\
        R_0(\ref{t7}):~                  & A \hand B \his A \hand B   \label{t8} \\
        R_\top(\ref{t7}):~               & A \hand B \his \top        \label{t9} \\
        R^{-}_\hand(\ref{t8}):~          & A \hand B \his A           \label{t10} \\
        R^-_\hand(\ref{t8}):~            & A \hand B \his B           \label{t11} \\
        R_\his(\ref{ax2}, \ref{t10}):~   & A \hand B \his \hsome{s}{A \hand B} \label{t12} \\
        R_\exists(\ref{t12}, \ref{t9}):~ & A \hand B \his \hsome{s}{\top}      \label{t13} \\
        R_\exists(\ref{t6}, \ref{t9}):~  & \hx \his \hsome{s}{\top}            \label{t14} \\
        R_{\op{init}}(\ref{t13}):~       & \op{init}(\top)                     \label{t15} \\
        R_0(\ref{t15}):~                 & \top \his \top \\
    \end{align}
    Полученное множетсво не содержит формулы $\hx \his C$. \\
    Следовательно, $\mathcal{O} \nvDash \hx \his C$.

    \begin{figure}[H]
        \centering
        \begin{prooftree} %[rule code={\hbox{\tikz \draw (0em,0) -- (\hsize,0);}}] % You should refer to ebproof's manual to twickle lines.
                        \Infer0[]{\dots}
                        \Infer1[]{A \hand \hsome{r}{B} \hisb A \hand \hsome{r}{B}}
                        \Infer1[$R^-_\hand$]{A \hand \hsome{r}{B} \hisb A}
                        \Infer1[$R_\his(2)$]{A \hand \hsome{r}{B} \hisb \hsome{s}{A \hand B}}

                        \Infer0[]{\dots}
                        \Infer1[$R_\his(2)$]{A \hand \hsome{r}{B} \hisb \hsome{s}{A \hand B}}
                        \Infer1[$R_{\op{init}}$]{\op{init}(A \hand B)}
                        \Infer1[$R_\top$]{A \hand B \hisb \top}

                    \Infer2[$R_\exists$]{A \hand \hsome{r}{B} \hisb \hsome{s}{\top}}

                    \Infer0[]{\dots}
                    \Infer1[$ $]{\hsome{s}{\top} \hisb \top}
                \Infer2[$R_\exists$]{A \hand \hsome{r}{B} \hisb \hsome{s}{\hsome{s}{\top}}}
                \Infer1[$R_\his(3)$]{A \hand \hsome{r}{B} \hisb B}

        \end{prooftree}
        \begin{prooftree}
                \Infer0[]{\dots}
                \Infer1[$R_\his(3)$]{A \hand \hsome{r}{B} \hisb B}
                    \Infer0[]{\op{init}(A \hand \hsome{r}{B})}
                    \Infer1[$R_0$]{A \hand \hsome{r}{B} \hisb A \hand \hsome{r}{B}}
                    \Infer1[$R^-_\hand$]{A \hand \hsome{r}{B} \hisb \hsome{r}{B}}

                    \Infer0[]{\op{init}(A \hand \hsome{r}{B})}
                    \Infer1[$R_0$]{A \hand \hsome{r}{B} \hisb A \hand \hsome{r}{B}}
                    \Infer1[$R^-_\hand$]{A \hand \hsome{r}{B} \hisb \hsome{r}{B}}
                    \Infer1[$R_{\op{init}}$]{\op{init}(B)}
                    \Infer1[$R_\top$]{B \hisb \top}

                \Infer2[$R_\exists$]{A \hand \hsome{r}{B} \hisb \hsome{r}{\top}}

            \Infer2[$R^+_\hand$]{A \hand \hsome{r}{B} \hisb B \hand \hsome{r}{\top}}
            \Infer1[$R_\his(2)$]{A \hand \hsome{r}{B} \hisb C}
        \end{prooftree}
    \end{figure}
\end{document}