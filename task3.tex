\documentclass[11pt]{article}
\usepackage{amsmath,amsthm,amssymb,amsbsy}
\usepackage{cmap}
\usepackage[utf8]{inputenc}
\usepackage[english,russian]{babel}
\usepackage{svg}
\usepackage{listings}
\usepackage{enumitem}
\usepackage{graphicx}
\usepackage{wrapfig}
\usepackage{hyperref}
\usepackage{caption,subcaption}
\usepackage{float}
\usepackage{pdfpages}
\usepackage[nottoc,numbib]{tocbibind}
\usepackage{tabularx}
\usepackage[T1]{fontenc}
\usepackage{tikz}
\usepackage{standalone}
\usepackage{ebproof}


\renewcommand{\baselinestretch}{1.5}
\newcommand{\rulesep}{\unskip\ \vrule\ }

%\renewcommand{\#}{C{\lserif\#}}

\newcommand{\code}[1]{\lstinline$#1$}
%\newcommand{\code}[1]{#1}

\newcommand{\op}[1]{\operatorname{#1}}
\newcommand{\sonop}[4]{#1~\leftarrow~\mathbf{#2}[#3](#4)}
\newcommand{\sonopa}[4]{#1~\leftarrow&~\mathbf{#2}[#3](#4)}

\lstset{captionpos=bellow}
\lstset{basicstyle=\small\ttfamily,showspaces=false,showstringspaces=false}

\theoremstyle{definition}
\newtheorem{definition}{Определение}

\theoremstyle{lemma}
\newtheorem{lemma}{Лемма}

\theoremstyle{statement}
\newtheorem{statement}{Утверждение}

\renewcommand\qedsymbol{QED}

\newlength{\myleftlen}
\newcommand{\setmyleftlen}[1]{\settowidth{\myleftlen}{\( \displaystyle
#1\)}}
\newcommand{\backup}{\hskip-\myleftlen\mkern-7mu}

\tikzset{>=latex}

\newcommand{\hsomet}[2]{\hsome{\textit{#1}}{\textbf{#2}}}
\newcommand{\hsome}[2]{\exists#1.#2}
\newcommand{\honlyt}[2]{\honly{\textit{#1}}{\textbf{#2}}}
\newcommand{\honly}[2]{\forall#1.#2}
\newcommand{\his}{\sqsubseteq}
\newcommand{\hisb}{~{\color{red}\boldsymbol{\sqsubseteq}}~}
\newcommand{\hand}{\sqcap}
\newcommand{\hor}{\sqcup}

\title{Решения задач по материалам лекции 4}
\author{Алексей Глушко}

\begin{document}
\maketitle
\section*{Задача 1}
    Построим замыкаение $\op{init}(A)$ относительно правил вывода.
    \begin{align}
        & \op{init}(A) \tag{0} \label{0} \\
        R_0(\ref{0}):~               & A \his^{\{\}} A           \label{1} \\
        R_\his(ax1, \ref{1}):~       & A \his^{\{\}} \hsome{r}{A_1} \hand \hsome{s}{A_1}           \label{2} \\
        R^-_\hand(\ref{2}):~         & A \his^{\{\}} \hsome{r}{A_1}                                \label{3} \\
        R^-_\hand(\ref{2}):~         & A \his^{\{\}} \hsome{s}{A_1}                                \label{4} \\
        R_{init}(\ref{3}):~          & \op{init}(A_1)                                              \label{5} \\
        R_0(\ref{5}):~               & A_1 \his^{\{\}} A_1                                         \label{6} \\
        R_\top(\ref{5}):~            & A_1 \his^{\{\}} \top                                        \label{7} \\
        R_\his(ax2, \ref{7}):~       & A_1 \his^{\{\}} B_1 \hand B_2                               \label{8} \\
        R^-_\hand(\ref{8}):~         & A_1 \his^{\{B_1\}} B_1                                      \label{9} \\
        R^-_\hand(\ref{8}):~         & A_1 \his^{\{B_2\}} B_2                                      \label{10} \\
        R_\his(ax3, \ref{9}):~       & A_1 \his^{\{B_1\}} C                                        \label{11} \\
        R_\his(ax3, \ref{10}):~      & A_1 \his^{\{B_2\}} C                                        \label{12} \\
        R_\exists(\ref{3},\ref{11}):~& A \his^{\{\hsome{r}{B_1}\}} \hsome{r}{C}                    \label{13} \\
        R_\exists(\ref{3},\ref{12}):~& A \his^{\{\hsome{r}{B_2}\}} \hsome{r}{C}                    \label{14} \\
        R_\exists(\ref{4},\ref{11}):~& A \his^{\{\hsome{s}{B_1}\}} \hsome{s}{C}                    \label{15} \\
        R_\exists(\ref{4},\ref{12}):~& A \his^{\{\hsome{s}{B_2}\}} \hsome{s}{C}                    \label{16} \\
        R^+_\hand(\ref{13},\ref{14},\ref{15},\ref{16}):~& A \his^{\{\hsome{r}{B_{1/2}} \hand \hsome{s}{B_{1/2}}\}}   \label{17} \\
        R_\his(ax5, \ref{17}):~      & A \his^{\{\hsome{r}{B_{1/2}} \hand \hsome{s}{B_{1/2}}\}} A  \label{18}
    \end{align}

    Wanted definitions of $A$ in $\mathcal{O}$:
    \begin{gather*}
        \hsome{r}{B_1} \hand \hsome{s}{B_1} \\
        \hsome{r}{B_1} \hand \hsome{s}{B_2} \\
        \hsome{r}{B_2} \hand \hsome{s}{B_1} \\
        \hsome{r}{B_2} \hand \hsome{s}{B_2}
    \end{gather*}

%\pagebreak
\section*{Задача 2}
    \begin{figure}[H]
        \centering
        \begin{tikzpicture}
            \node[circle, draw, label={$A^{\mathcal{I}}$}] (ai) at (-4, 0) {};
            \node[circle, draw, label={[align=center]$A_1$,\\$B^{\mathcal{I}}_1, B^{\mathcal{I}}_2, C^{\mathcal{I}}$,\\$B^{\mathcal{J}}_1, B^{\mathcal{J}}_2, C^{\mathcal{J}}$}] (a1) at (0, 0) {};
            \node[circle, draw, label={$A^{\mathcal{J}}$}] (aj) at (4, 0) {};

            \draw[->] (ai) to node[midway,above] {r} (a1);
            \draw[->] (aj) to node[midway,above] {r} (a1);
            \draw[->] (ai) to[out=-45, in=-135] node[midway,below] {$s^{\mathcal{I}}$} (a1);
            \draw[->] (aj) to[out=-135, in=-45] node[midway,below] {$s^{\mathcal{J}}$} (a1);
        \end{tikzpicture}
    \end{figure}

%\pagebreak
\section*{Задача 3}
    \dots

\end{document}